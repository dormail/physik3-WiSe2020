\documentclass{article}

\usepackage{amsmath, amssymb}
\usepackage[margin=2.5cm]{geometry}

\begin{document}

\section*{Aufgabe 3}

\subsection*{a)}

Das System besteht aus 3 Massen, welche unter der Vorraussetzung,
dass diese Punktartig sind, 9 Freiheitsgrade besitzen.

\noindent
Diese werden durch folgende Zwangsbedingungen eingeschränkt:
\begin{itemize}
	\item $m_2$ bewegt sich nur auf z-Achse 
		$\Rightarrow x_2 = y_2$
		\\
		$\Rightarrow$ Holonome, skleronome Zwangsbedingung.
	\item z-Koordinaten der Massen durch $\theta$ vorgegeben
		\[
			z_1 = R \cdot \cos\theta,
			\quad
			z_2 = 2R \cdot \cos\theta
		\]
		$\Rightarrow$ Holonome, rheonome Zwangsbedingung (da $\theta$
		zeitabhängig sein kann)
	\item Abstand der Massen $m_1$ zur Rotationsachse durch $\theta$ 
		vorgegeben
		\[
			r_\perp = R \cdot \sin\theta
		\]
		$\Rightarrow$ Holonome, rheonome Zwangsbedingung (da $\theta$
		zeitabhängig sein kann)
	\item Ausrichtung der $m_1$ durch Rotation $\varphi := \omega t$ eingeschränkt
		\[
			\vec m_1 = R \cdot cos\theta \cdot \vec e_z
			\pm r_\perp \cdot
			\begin{pmatrix} \cos\varphi \\ \sin\varphi \\ 0 \end{pmatrix}
			% darstellung alles in einem vektor
			= R \cdot 
			\begin{pmatrix}
				\pm \sin\theta \cos(\omega t) \\
				\pm \sin\theta \sin(\omega t) \\
				\cos\theta
			\end{pmatrix}
		\]
		$\Rightarrow$ Holonome, rheonome Zwangsbedingung.
\end{itemize}

Somit lauten die Ortsvektoren:
\[
	\vec r_1
			= R \cdot 
			\begin{pmatrix}
				\pm \sin\theta \cos(\omega t) \\
				\pm \sin\theta \sin(\omega t) \\
				\cos\theta
			\end{pmatrix}
	\qquad
	\vec r_2 
			= \cdot 
			\begin{pmatrix}
				0 \\
				0 \\
				2R \cdot  \cos\theta
			\end{pmatrix}
\]

\end{document}
